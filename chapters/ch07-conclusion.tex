% !TEX root = ../thesis.tex
\chapter{Conclusion}
%TODO continous action space for threshold
In conclusion, in this work, we managed to build a framework for the training and testing of various deep reinforcement learning algorithms inside polyhedra analysis. We could then design a series of such algorithms. These algorithms ranged from very global reinforcement learning methods to ones more optimised with problem-specific knowledge as well as having a greater flexibility towards the various subproblems. We then tested out the different subparts of our algorithms we had come up with, in the goal of finding an optimal combination.\\
Once our algorithms were designed, we could then test them on a broad array of different benchmarks. They outperformed other preexisting reinforcement learning methods on both accuracy and performance.\\
It is also worth noting that we highlighted the overall effectiveness of the global deep Q-network algorithm, as, after all, it was able to craft a decision policy that was the most accurate. Even with the design of new training algorithms that used more domain-specific knowledge, we could only increase the performance at the loss of some precision.\\
Future research for this work could be done in several directions. Techniques such as CNN's could be investigated for automatic feature extraction from the polyhedra itself. Other works such as \cite{dyer1991random, kim2004fast} could also be used for the approximation of the volume of the polyhedra. Both techniques could then be simultaneously used for expanding the features as well as the reward for precision. Other than that, optimisations to the neural network itself could be undertaken as this was not explored much in this work. An expansion of the action set, most notably the size and number of different thresholds, perhaps even using a continuous action set for the size of the threshold could increase the precision. Finally, separating the features, NN's and making them more task-specific, would increase the overall performance.