% !TEX root = ../thesis.tex

% set counter to n-1:
\setcounter{chapter}{0}

\chapter{Introduction}
%TODO insert examples of bugs
As technology becomes ever more present in the modern age and tasks become more and more optimised, the more the structures in our lives become controlled by programs. From simple things as automatic doors, too self-driving metros, trains and even cars, to medical software, aviation software and even nuclear weapon software. The more we wish to automise and simplify our lives the bigger will be the power that we will put into the hands of computers and the programs that run on them. As these programs get bigger, writing them becomes more complicated they become longer and the risk of them containing errors increases. However, the safety and invulnerability of some of these systems is critical and needs to be verifiable. Static analysis is a sub-branch of computer science tasked with the analysis of computer programs without actually executing them. It has seen a growing commercial use in the past years in some of these safety-critical domains. 

\section{Problem Statement}

Unfortunately, the complexity and size of the programs being used today has grown drastically and the design of a static analyser that can keep up with this growth is not a simple task. Many techniques exist that exploit specifics about the analyses that mange to increase its precision. There are also different types of techniques that try to leverage precision loss against performance gain. However, finding the right balance between these two is not an easy task.
 

\section{Goals}

The goal of this thesis is to incorporate advanced reinforcement learning techniques inside polyhedra analysis. The goal is then to optimise these methods in order to render the analysis more efficient and outperform other analysis methods.


\section{Structure of this Document}.

The remaining section of this chapter will introduce some conventions made for the sake of readability and briefness of the complete document.

Chapter 2 will describe reinforcement learning and talk about the recent advances in this field. I will as well describe deep Q-networks and some of areas that they have been used.

Chapter 3 addresses Polyhedra analysis and some of the methods used to make it more efficient. 

Chapter 4 Will address the algorithm designed and how the two previous chapters were combined, in order to achieve polyhedra analysis with deep Q-networks.

%TODO finish chapters results etc...

%\section{Conventions}
%
%Throughout this document, we will adhere the following conventions:
%
%\begin{itemize}
%\item To reduce redundancy and increase readability, we will refer to 'the user' (or other third parties) as a male person. We do not intend to discriminate anybody.
%\item 
%\item 
%\end{itemize}

